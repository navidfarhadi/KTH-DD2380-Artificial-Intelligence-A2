\documentclass[12pt]{article}

% Packages
\usepackage[english]{babel}
\usepackage{natbib}
\usepackage{url}
\usepackage[table]{xcolor}
\usepackage[utf8x]{inputenc}
\usepackage{amsmath}
\usepackage{parskip}
\usepackage{fancyhdr}
\usepackage{vmargin}
\usepackage{graphicx}
\usepackage{hyperref}
\usepackage{array}
\usepackage{amssymb}
\usepackage{pgfgantt}
\usepackage{tikz}
%\usepackage{pgfplots}
\usepackage{algpseudocode}
\usepackage{algorithm}
%\usepackage[ngerman]{datetime}

% Mark the meta information
\title{Questions for Assignment 2}
\author{Franz Fuchs \& Navid Farhadi}																
\date{8 Mai 2015}

% Overwrite the english settings with german 
\addto\captionsenglish {
  \renewcommand{\contentsname}{Inhalt}
}
%\newdateformat{germanformat}{\THEDAY{. }\monthnamengerman[\THEMONTH], \THEYEAR}

% Set the layout
\makeatletter
\let\thetitle\@title
\let\theauthor\@author
\let\thedate\@date
\makeatother
\graphicspath{{../../images/}}
\setmarginsrb{3 cm}{2.5 cm}{3 cm}{2.5 cm}{1 cm}{1.5 cm}{1 cm}{1.5 cm}


% Mark the headers and footers
\pagestyle{fancy}
\fancyhf{}
\rhead{\theauthor}
\lhead{\thetitle}
\cfoot{\thepage}

% reformat tables
\newcolumntype{M}[1]{>{\centering\arraybackslash}m{#1}}
\newcolumntype{N}{@{}m{0pt}@{}}

% Remove the ugly borders around the hyperlinks
\hypersetup{
    colorlinks,
    linkcolor={blue!50!black},
    citecolor={blue!50!black},
    urlcolor={blue!80!black}
}

\newenvironment{equationate}{%
 \itemize
 \let\orig@item\item
 \def\item{\orig@item[]\refstepcounter{equation}\def\item{\hfill(\theequation)\orig@item[]\refstepcounter{equation}}}
}{%
 \hfill(\theequation)%
 \enditemize
}

\begin{document}

%%%%%%%%%%%%%%%%%%%%%%%%%%%%%% Cover Page %%%%%%%%%%%%%%%%%%%%%%%%%%%%%%%%%%%%%%%

\begin{titlepage}
	\centering
	\includegraphics[scale = 0.5]{KTH_Logo.jpg}\\[1.0 cm]
    \textsc{\LARGE KTH Royal Institute of Technology}\\[0.25 cm]
    \textsc{School of Electrical Engineering and Computer Science}\\[1.0 cm]
	
	\rule{\linewidth}{0.2 mm} \\[0.4 cm]
	{ \huge \bfseries \thetitle}\\
	\rule{\linewidth}{0.2 mm} \\[1.5 cm]
	
	
	

	\today\\[0.10 cm]
	\textsc{Autumn Term 2018}
\end{titlepage}

%%%%%%%%%%%%%%%%%%%%%%%%%%%%%%%% Table of contents %%%%%%%%%%%%%%%%%%%%%%%%%%%%%%%%%%%%%%

\pagebreak

%%%%%%%%%%%%%%%%%%%%%%%%%%%%%%%%%%%%%%%%%%%%%%%%%%%%%%%%%%%%%%%%%%%%%%%%%%%%%%%%%%%%%%%%%

\section*{Describe the possible states, initial state, transition function}

Possible states: All possibilities how player A and B could have made their moves (where the crosses and circles are on the 3 $\times$ 3 board)

Initial state: An empty 3 $\times$ 3 board

Transition function: The function that maps each state to a set of states the are achievable after the move of one of the players. (in other words: how will the board look after either player A or player B made a move)


\section*{Describe the terminal states of both checkers and tic-tac-toe}

For checkers: terminal states are a set of state which come follow from the
following situation: one player won (which means that the other player lost).
A winning state is state where only pieces of the winning player are on the field (between 1 and 12 pieces)

For tic-tac-toe: The game is finished when one player has won or when a tie was reached.
A player has won if he managed to reach 3 pieces in a row (vertically, diagonally, and horizontally).
A tie was reached when all fields are filled but no player could reach the goal of having three
pieces in a row.

\section*{Why is $v(A,s)$ a valid heuristic function for checkers}

The function $v(A,s) = \#(white-checkers) - \#(red-checkers)$ is a valid heuristic
function because it somehow says how good it is for a player to be a state. The amount
of red checkers in comparison to the white checkers says how many possible moves a player
can make. In general, the more moves a player can make the better are its chances of winning
because the proability for more powerful moves is greater. Therefore, a positive number for player A
says that it has more checkers on the board, which increases its chances of winning.

\section*{When does $v$ best approximate the utility function, and why}

If $s$ is a winning state, the heuristic function $v(A,s)$ approximates the utility function
best because then a positive number always leads to a win of player A. (A winning state of player A means that
only its checkers are on the field which leads to a positive number)

\section*{Can you provide an example of a state $s$ where $v(A,s) > 0$ and B wins in the following turn}

For example, $s$ is a state where two white checkers and one red checker is on the field. If player A enters this state, it will
be the turn of player B. If the the white checkers are aligned in a row and one free field in between them the red checker
can jump over the first white checker and subsequently over the other white checker, which means that B has won.

\section*{Will $\eta$ suffer from the same problem (referred to in the last question) as the evaluation function $v$}

No, it will not suffer from this problem. Player A will detect that player B can reach a terminal state
in which player B wins. Therefore, player A will not choose to make this move as described in the previous
question.


\end{document}
